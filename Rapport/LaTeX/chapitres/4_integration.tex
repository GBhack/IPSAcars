 \subsection{Circuit}


\subsection{Robot}

	\subsubsection{Intégration Matérielle}

		\paragraph{Ensembles propulsifs}

			\textit{[Motoréducteur + roue + encodeur]}

		\paragraph{Carte Réflecteurs Optiques}\label{integrationReflecteurs}



			\textit{[Espacement des capteurs + hauteur sur piste]}




			Nous connecterons la carte au moyen d'une nappe HE10 à 10 connecteurs.\\

			Le schéma électronique et les gerbers du circuit imprimés sont disponibles en annexe \ref{schemasCarteReflecteurs} (page \pageref{schemasCarteReflecteurs})

		\paragraph{Carte-Mère}\label{carteMere}


		\paragraph{Structure du robot}


			\textit{[Profilés alu standards, vis idem...]}


		\paragraph{Intégration globale, maquette numérique}

			\textit{[Vues Catia et breve explication]}

	\subsubsection{Intégration Logicielle}

\begin{itshape}
Modules indépendants à 4 niveaux (acquisition, traitement, décision, action), les sorties des uns servant d'entrées aux autres (mux) communicant via socket UDP => possibilité de faire interagir des programmes en C, en Python, en Java... 

+ un programme "lanceur et ordonnanceur" avec son fichier de config. Interface web pour programmation et paramétrage.

Modules "de base" livrés sous forme de classes avec nombreuses méthodes fournies.

IMPORTANCE DE LA DOCUMENTATION

\end{itshape}