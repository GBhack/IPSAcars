Ce dossier constitue l'étude préliminaire d'un projet de réalisation de plateforme d'enseignement et de recherche pour le pôle [NOM ??] de l'IPSA, école d'ingénieur de l'air et de l'espace.
Cette plateforme, composée d'un "circuit" interactif modelisant un réseau routier urbain et de "robots" autonomes capables de se déplacer en autonomie et en respect des règles de la route sur ce circuit, pourra servir de support de travaux pratiques pour l'ensemble des enseignements du pôle, voir même de support de recherche.
Le présent dossier propose une analyse détaillée de ce besoin et de ses implications, ainsi que la description d'une solution technique complète et argumentée. Enfin, une estimation des coûts et des plannings est offerte.
Ce dossier est agrémenté de diverses illustrations, schémas, plans et vues 3D, formant ainsi un guide presque suffisant pour la réalisation du projet.