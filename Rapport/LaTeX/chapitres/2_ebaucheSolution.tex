Nous pouvons d'ores et déjà ébaucher les diagrammes des cas d'utilisation de haut niveau suivants :

\subsection{Pour le robot :}

	\begin{tikzpicture} 
		\begin{umlsystem}[x=10, fill=blue!10]{Robot} 
		\umlusecase[fill=white]{Charger un programme} 
		\umlusecase[y=-2,fill=white]{Evoluer en autonomie} 
		\umlusecase[y=-4,fill=white]{Changer la batterie} 
		\end{umlsystem} 
		 
		\umlactor{Développeur}  
		\umlactor[y=-3]{Utilisateur} 
		   
		\umlinherit{Développeur}{Utilisateur} 
		\umlassoc{Développeur}{usecase-1} 
		\umlassoc{Utilisateur}{usecase-2} 
		\umlassoc{Utilisateur}{usecase-3}	 
	\end{tikzpicture} 

	\textbf{Reprogrammation :}
	\begin{itemize}
		\item \textbf{Précondition :} Nous disposons d'un programme fonctionnel.
		\item \textbf{Déclencheur :} Un développeur souhaite implémenter un programme.
		\item \textbf{Scenario :}
		\begin{enumerate}
			\item On connecte le robot à une source de tension.
			\item On établit une liaison entre le robot et un ordinateur.
			\item Le développeur lance le téléversement du programme.
			\item Le robot acquitte.
			\item On ferme la connexion.
		\end{enumerate}
	\end{itemize}

	\textbf{Evolution en autonomie:}
	\begin{itemize}
		\item \textbf{Précondition :} Le robot a été programmé et dispose d'une batterie chargée.
		\item \textbf{Déclencheur :} Un membre du laboratoire souhaite observer le comportement du robot.
		\item \textbf{Scenario :}
		\begin{enumerate}
			\item On met le robot sous tension.
			\item On place le robot sur une ligne blanche du circuit.
			\item On appuie sur le bouton de démarrage de séquence
			\item Le robot déclenche les programmes en mémoire.
		\end{enumerate}
	\end{itemize}

	\textbf{Changement de batterie:}
	\begin{itemize}
		\item \textbf{Déclencheur :} On souhaite changer la batterie du robot
		\item \textbf{Scenario :}
		\begin{enumerate}
			\item On met le robot hors tension.
			\item On accède à la batterie en place le cas échéant, et on la retire.
			\item On met en place la nouvelle batterie
			\item On remet en place les éléments éventuellement retirés pour accéder à la batterie.
		\end{enumerate}
	\end{itemize}


	Le mode d'utilisation primaire est bien évidemment celui de l'évolution en autonomie. Le robot étant destiné à servir de plateforme de recherche, et devant donc être entièrement reprogrammable, il est délicat de décrire ce mode d'utilisation qui dépendra intégralement du programme chargé par l'utilisateur.\\

	Nous nous appliquerons cependant à décrire le mode d'utilisation correspondant à l'application la plus basique du robot (et de fait au programme que nous livrerons avec ce dernier).

	
\subsection{Pour le circuit :}

	\begin{tikzpicture} 
		\setcounter{tikzumlUseCaseNum}{0}
		\begin{umlsystem}[x=10, fill=blue!10]{Circuit} 
		\umlusecase[fill=white]{Charger un programme} 
		\umlusecase[y=-2,fill=white]{Evoluer en autonomie}  
		\end{umlsystem} 
		 
		\umlactor[y=0.5]{Développeur}  
		\umlactor[y=-2.5]{Utilisateur} 
		   
		\umlinherit{Développeur}{Utilisateur} 
		\umlassoc{Développeur}{usecase-1} 
		\umlassoc{Utilisateur}{usecase-2}  
	\end{tikzpicture}