La voiture autonome, loin du concept de science-fiction qu'elle pouvait représenter il y a quelques années est en train de devenir une réalité.\\

Si cette ambition put être à une certaine époque motivée par le simple attrait de la prouesse technique, nous percevons aujourd'hui tous les bénéfices que l'on pourrait en tirer.
En effet, les avancées scientifiques et techniques nous permettent désormais de prétendre à concevoir une voiture qui soit non seulement autonome, mais surtout intélligente. Il est aujourd'hui tout à fait réaliste de penser que dans les quelques années à venir les voitures sauront adopter un comportement bien plus intélligent que celui de leurs conducteurs actuels, et ce au profit de la sécurité, de l' efficience énergétique mais également de l'encombrement des axes routiers.\\

A terme, nous pouvons facilement imaginer que les différents véhicules auront la possibilité de communiquer entre eux afin de prévenir les véhicules environants de leurs intentions, mais celà ne les dispensera pas de devoir être capables d'évaluer leur environnement afin d'y détecter les éléments "indépendants" (piétons, obstacles...). \\

Comme toute révolution technologique, la voiture intelligente devra faire face au caractère progressif de son adoption: toutes les voitures sur les routes ne deviendront pas autonomes du jour au lendemain. Ces véhicules devront donc également être capables d'évoluer au milieu d'une circulation telle que nous la connaissons, où chaque acteur adopte un comportement presque parfaitement aléatoire, et ne signale pas toujours ces intentions.\\

Afin d'ajouter une dimension supplémentaire à ce projet, nous avons souhaiter apporter une intélligence propre aux feux de circulations eux-mêmes. Ainsi, les feux adopteraient un comportement en fonction du trafic. Ceci s'inscrit également dans une démarche d'optimisation de la circulation, et il est très réaliste d'espérer que cette technologie déjà existante se fera omniprésente dans les années qui viennent, d'où notre volonté d'inclure cet élément d'environement à notre projet.\\

Le but de ce projet est donc d'étudier notre capacité à faire cohabiter intélligence artificielle et envronnement "réel" et indépendant avec des moyens techniques et financiers extremement restreints, mais également et surtout de fournir une plateforme d'expérimentation aux étudiants et chercheurs.