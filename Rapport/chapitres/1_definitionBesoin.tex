Une définition pertinente du besoin est une condition absolument nécéssaire pour la bonne réalisation de tout projet. Nous nous emploierons donc à définir aussi précisément et pertinement que possible le besoin motivant ce projet, et à régulièrement revenir sur ce dernier afin de prendre en compte d'éventuelles évolutions et à prévenir toute dérive du projet.

\subsection{Besoin}
Bénéficier d'une plate-forme de développement modulaire et évolutive permettant d'implémenter des algorithmes d'automatismes dans le cadre de recherches liées aux voitures dites "intelligentes" et d’optimisation de la gestion du trafic.

\subsection{Exigences}
\begin{itemize}
	\item La plate forme devra fournir des véhicules autonomes (et sans fils), chaque véhicule devant être équipé:
	\begin{itemize}
		\item D'une caméra.
		\item De capteurs optiques permettant un "suivi de ligne" au sol.
		\item D'un ordinateur embarqué, de puissance de calcul suffisante pour assurer le traitement d'image en "quasi-temps réel" (~10 images/seconde) et le contrôle du robot.
		\item De deux roues de propulsion commandées par un moteur CC.
		\item D'une roue directrice commandée par un servo-moteur.
		\item De "clignotants" à LED sur la face avant ET les flancs du robot ainsi que d'un "feu stop" à l'arrière.
		\item D'un capteur de distance à l'avant du robot.
		\item D'une autonomie d'au moins 15 minutes.
		\item D'un bouton de démarrage de séquence facilement accessible.
	\end{itemize}
	\item La plate-forme sera également constituée d'un "circuit" répondant lui même aux exigences suivantes :
	\begin{itemize}
		\item Présence d'au moins un carrefour à feux bicolores (commandés par une carte FPGA)
		\item Présence de capteurs de présence de véhicules aux abords des feux.
		\item Présence de lignes blanches (sur fond noir) au sol permettant le guidage des robots.
		\item Présence de "parois" au bord des "routes" permettant une isolation visuelle du circuit.
	\end{itemize}
\end{itemize}